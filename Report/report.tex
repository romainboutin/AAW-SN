\documentclass[12pt,a4paper,titlepage]{article}
% enlever twoside pour la version pdf
\usepackage[a4paper,hmarginratio=2:2]{geometry}
\usepackage[french]{babel}
\usepackage[utf8]{inputenc}
\usepackage[T1]{fontenc}
\usepackage{lmodern}
\usepackage{epsfig}
\usepackage{wrapfig}
\usepackage{thmbox} % theorem box (joli theoreme)
% \usepackage{amsthm} % theorem
\usepackage{setspace} % theorem interline
\usepackage{amssymb}
\usepackage{amsmath}
\usepackage{alltt}
\usepackage[backref=page,plainpages = false]{hyperref}
\usepackage{parskip}
\usepackage{verbatim}
\usepackage{listings}
%\usepackage{marvosym}
\usepackage{fancyhdr}
\usepackage{color}

\pagestyle{fancy}
\fancyhead[LE,RO]{\thepage}
\fancyfoot{}

% define custom color
\definecolor{red75}{rgb}{0.75,0,0}
\definecolor{blue50}{rgb}{0,0,0.5}

% theorem list
% \newtheorem{mydef}{Definition}

\renewcommand{\labelitemi}{\textbullet}
\setlength{\parindent}{0cm} % pas d'alinéa





\title{Rapport du projet d'AAW :\\ 
Réseau social en Java J2EE.}
\author{William Le Coroller \& Romain Boutin}
\date{Mercredi 25 Février 2015}

\begin{document}
\pagenumbering{Roman}



% - - - - - - - début de la page de garde - - - - - - - - - - - - - - - - - - -



\thispagestyle{empty}

\maketitle

\clearpage{\pagestyle{empty}}



% - - - - - - - fin de la page de garde - - - - - - - - - - - - - - - - - - - -



\clearpage{\pagestyle{empty}}

\tableofcontents
\clearpage{\pagestyle{empty}}

\pagenumbering{arabic}



% ========== nouvelle page ====================================================



\clearpage{\pagestyle{empty}}



\section{Introduction}

Le but de ce projet est la réalisation d'un réseau social en utilisant les technologies
Java J2EE.\\

Le rapport est organisé en quatres parties principales :
\begin{itemize}
\item La partie \begin{bf}contexte\end{bf} qui présente les notions et définition
de la théorie des graphes ainsi que ce qui est attendu ainsi que les contraintes.
\item La partie \begin{bf}théorie\end{bf} qui explicite les deux algorithmes de 
calcul des composantes fortement connexes et qui les compare très succintement.
\item La partie \begin{bf}technique \& documentation\end{bf} qui présente 
les choix techniques qui ont été effectué et fait le lien avec la documentation.
\item La partie \begin{bf}résultats\end{bf} qui présente 
les résultats des brefs tests de performances effectué sur les deux algorithmes
implémentés.
\end{itemize}



% ========== nouvelle page ====================================================



\clearpage{\pagestyle{empty}}



\section{Choix techniques \& documentations}
La documentation du projet est extrait des commentaires du code source puis 
formatté par doxygen pour le rendre plus lisible. Il est disponible dans le 
document PDF \LaTeX ci-joint (document \og{}refman.pdf\fg{}). Il est également disponible 
sous un format web HTML \og{}/doc/doxygen/html/index.html\fg{}.

Durant l'implémentation de ce projet, plusieurs choix techniques ont été fait. 
Nous tentons ici des les présenter et de les justifier lorsque cela est nécéssaire.

Le premier choix qui a été fait fut d'\begin{bf}utiliser le langage C/C++\end{bf}. 
En effet, le Java aurait certainement été plus simple et plus rapide d'utilisation. 
Néanmoins en utilisant le C/C++ on avait déjà une bibliothèque de gestion des graphes toutes 
faites nous permettant un gain de temps substancielle.

Le second choix qui a été fait est de donner la possibilité à l'utilisateur de 
\begin{bf}sauvegarder le graphe généré aléatoirement dans un fichier de description\end{bf} afin 
de pouvoir le re-entrer pour re-tester le graphe généré à plusieurs reprises.
En effet, n'ayant pas implémenter la fonctionnalité d'input demandé, on a 
voulu donner à l'utilisateur la possibilité de re-tester le graphe généré.

Le troisième choix qui a été fait est d'\begin{bf}utiliser des \og{}vector\fg{}\end{bf} de la bibliothèque 
standard C/C++ afin de représenter les arêtes du graphe sous la forme d'une liste
d'adjacence plutôt qu'une classe liste modifiée. Dans un premier temps, nous 
avions réalisé une classe implémentant la structure de donnée d'une liste 
chainée en utilisant des templates. 
Le soucis majeur de cette implémentation est qu'une telle structure ne 
pouvaient être détruite facilement dès lors qu'on lui passait une classe 
en type patron.
N'ayant pas trouvé de solution \begin{bf}simple\end{bf}, nous avons décider de
nous en remettre à quelque chose de plus sûr. Les fichiers implémentant cette 
classe sont présent dans le dossier \og{}/src/\fg{} bien que non répertorié par 
doxygen.

Enfin, le quatrième et dernier choix est vis à vis de l'implémentation des algorithmes. 
\begin{bf}On a choisi d'implémenter les deux algorithmes\end{bf} majeurs afin de pouvoir 
les comparer de façon grossière par curiosité. Les tests et résultats sont 
présentés dans la partie suivante.

On a, par ailleurs, vérifié en utilisant des outils externes tel que 
\begin{bf}gdb\end{bf} ou \begin{bf}valgrind\end{bf} 
qu'il n'y ai pas de fuite mémoire.



% ========== nouvelle page ====================================================



\clearpage{\pagestyle{empty}}



\section{Résultats}




% ========== nouvelle page ====================================================



\clearpage{\pagestyle{empty}}



\section{Conclusion}
Du point de vue des résultats, il apparaît que l'algorithme de Tarjan semble 
plus rapide que l'algorithme de Kosaraju-Sharir.

On regrette de ne pas avoir implémenter une \og{}vraie\fg{} liste ou 
une \og{}vraie\fg{} pile ceci bien que la structure vector fonctionne très bien 
(probablement beaucoup mieux que n'importe quel liste ou pile de mon cru).

On regrette également de ne pas avoir davantage étoffé les tests effectué. 
On aurait pu leur donner un semblant d'approche scientifique.

On est néanmoins satisfait d'avoir implémenté les deux algorithmes principaux 
de calcul de composantes fortement connexes d'un graphe orienté et d'avoir 
mis en place un système simple afin de comparer grossièrement leurs performances.



% ========== nouvelle page ====================================================



\clearpage{\pagestyle{empty}}



\end{document}